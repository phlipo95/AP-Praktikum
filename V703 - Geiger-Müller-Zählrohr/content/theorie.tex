\section{Theoretische Grundlage}
\label{sec:Theorie}
Ziel des Versuches ist es die Totzeit, die Nachentladung sowie Charaketristik eines Geiger-Müller-Zählrohrs zu bestimmen.


\subsection{Aufbau eines Geiger-Müller-Zählrohr}
Ein Geiger-Müller-Zählrohr ist ein Messgerät zur Bestimmung von Ionisiernder Strahlung. Ein Modell eines Geiger-Müller-Zählrohrs ist in Abbildung \ref{fig:skizze} zu sehen.
\begin{figure}
  \centering
  \includegraphics[height=4cm]{picture/Skizze.pdf}
  \caption{Aufbau eines Geigermüllerzählrohr \cite{sample}}
  \label{fig:skizze}
\end{figure}
Zwischen dem Anodendraht und dem Stahlmantel wird eine äußere Spannung angelgt, wodurch eine radialsymetrisches Feld zwischen Anode und Kathode entsteht, deren Feldstärke
\begin{equation}
  E(r) = \frac{U}{r\, \ln\left(\frac{r_\text{k}}{r_\text{a}}}\right)
  \label{eqn:feld}
\end{equation}
beträgt. Der Stahlzylinder ist mit einem Gasgemisch aus Argon und Etyhlalkohol gefüllt und wird mit einer Mylar-Folie verschlossen. Die Beimischung des Alkohols soll die Nachentladungen, auf welche im weiteren Verlauf noch weiter eingegangen wird, unterdrücken. Durch die Wahl von Mylar-Folie als Stirnfenster soll die Abschirmung von $\alpha$ verhindert werden und diese somit für das Geiger-Müllerzählrohr messbar sein.


\subsection{Wirkungsweise des Geiger-Müller-Zählrohrs bei unterschiedlichen Spannungen}
Beim eintreffen von ionisierter Strahlung in das Geiger-Müller-Zählrohr werden entlang der Teilchenbahn soviel Gasatome ionisiert, bis die Teilchenenergie der Größenordnung von 100\,keV in Ionisierungsenergie der Atome umgewandelt ist. Somit besteht eine Proportionalität zwischen der Energie der eintreffenden Strahlung und des gemessenen Stroms. Je nach angelegter Spannung können verschieden Effekte im Zählrohr beobachtet werden, wie in Abbildung \ref{fig:Geb} dargestellt ist.
\begin{figure}
  \centering
  \includegraphics[height=9cm]{picture/Gebiete.pdf}
  \caption{Messbereiche des Geiger-Müller-Zählrohrs bei unterschiedlichen Betriebsspannungen \cite{sample}}
  \label{fig:Geb}
\end{figure}
Wird nun eine sehr kleine Zählrohrspannung $U$ angelegt, rekombiniert sich ein großer Teil der Ionisierten Atome wieder, bevor die Elektronen den Draht erreichen (siehe Gebiet I). Wird die Zählrohrspannung weiter erhöht, kann ein Ionisationsstrom zwischen der Anode und der Kathode gemessen werden, welcher Proportional zur Energie, sowie auch der Intensität der Ionisierenden Strahlung ist. In Abbildung \ref{fig:Geb} ist dieses Gebiet mir einer roten II gekennzeichnet. Wird die Spannung weiter gesteigert, erhalten die freigesetzten Elektronen durch das Radialsymmetrische Feld so viel kinetische Energie das diese auf den Weg zur Anode durch Stöße zusätzliche Argon Atome ionisieren. Dieses Phönomen wird als Townsend-Lawine bezeichnet. Dieser Bereich (III) wird als Proportionalbereich bezeichnet und eignet sich besonders gut um die Teilchenenergie zu bestimmen, da ein hinreichend großer Strom gemessen werden kann, der proportional zur Teilchenenergie ist. Im Bereich IV wird die Spannung weiter erhöht, so dass die Elektronen beim Auftreffen auf die Argon Atome, diese Einerseits ionisieren als auch anregen. Die daraus entstehenden Photonen ionisieren daraufhin im Gegensatz zu den Elektronen nicht nur Atome in Feldrichtung sondern Atome im gesamenten Zylindermantel. Es kommt zu einer primären Elektronenlawine im gesamten Zählrohrvolumen. Dies hat zur Folge das keine Aussage mehr über die Energie der Strahlung getroffen werden kann, lediglich über die Intensität. Wenn die Kinetische Energie der Ionen groß genug wird lösen diese beim auftreffen auf den Zählrohrmantel sekundär Elektronen aus und es kommt zu einer Nachentladung. Auf den Aspekt der Nachentladung wird in den folgenden Kapiteln noch eingegangen.


\subsection{Totzeit}
Aufgrund der wesentlich größeren Masse der Ionen im Gegensatz zu den Elektronen ist deren Trägheit größer. Somit brauchen die Ionen eine wesentlich längere Zeit bis sie an der Kathode neutral geladen worden sind. In dieser Zeit bauen diese einen Ionenenschlauch auf, welcher die Feldstärke soweit herabsetzt das es in der Zeit $T$ nicht mehr zu Stoßionisation kommt. In dieser Zeit kann also keine Ionisierende Strahlung mehr nachgewiesen werden. Anschließend an die Totzeit sind die Elektrischen Impulse zunächst geringer als der Primärimpuls wie in Abbildung \ref{fig:tot} zu sehen ist. Die Zeit bis die kompletten Ionen abgebaut sind wird Erholungszeit $T_\text{E}$ genannt.
\begin{figure}
  \centering
  \includegraphics[height=5cm]{picture/Nachentladung.pdf}
  \caption{Tot- und Erholungszeit des Geigermüllerzahlrohrs \cite{sample}}
  \label{fig:tot}
\end{figure}


\subsection{Nachentladungen}
Wenn die Ionen durch das E-Feld hinreichend viel kinetische Energie gewinnen kann es sein, dass diese beim auftreffen auf die Mantelfäche wiederum Elektronen aus dieser Herauslösen. Dies ist Ausgangsimpuls von einer neuen Elektronenlawine. Um dies zu verhindern wird dem Gas, Alkohol beigemischt, auf welches die Ionen stoßen und so ein Teil ihrer Kinetischen Energie verlieren, um so mit einer niedrigeren Geschwindigkeit auf den Stahlmantel zu treffen und keine Elektronen mehr herraus zu lösen.


\subsection{Charakteristik des Zählrohrs}
Mit der Charakteristik eines Zählrohrs wird das Plateau gemeint, das sich ausbildet wenn die registrierte Teilchenzahl $N$ gegen die Zählrohrspannung $U$ bei konstanter Intensität aufgetragen wird. Ein Schema ist in Abbildung \ref{fig:Pla} zu sehen.
\begin{figure}
  \centering
  \includegraphics[height=5cm]{picture/Plateau.pdf}
  \caption{Charakteristik des Zählrohrs \cite{sample}}
  \label{fig:Pla}
\end{figure}
Aufgrund der Nachentladung weist der Graph eines Geiger-Müller-Zählrohrs im Gegensatz zum idealen eine Steigung des Plateaus auf, welche den Nachentladungen geschuldet ist. Eine geringe Steigung spricht für ein qualitativ hochwertiges Geiger-Müller-Zählrohr.


\subsection{Bestimmung der Totzeit mit der Zwei-Quellen-Methode}
Aufgrund der Totzeit des Geiger-Müller-Zählrohrs wird das Phänomen beobachtet, dass die Summe der Impulsraten von zwei Präperaten ($N_1 + N_2$) größer ist, als wenn beide Präperate ($N_{1+2}$) gleichzeitig auf das Geiger-Müller-Zählrohr gerichtet werden.
\begin{equation}
  	N_1 + N_2 > N_{1+2}
  \label{eqn:ungl}
\end{equation}
Wird nun für die aus der Messung dedektierten Impulse $N_\text{r}$ berücksichtig, dass die Zählrate $N_\text{w}$ für den Momment der Totzeit $T$ keine Strahlung registiert ergibt sich die Gleichung
\begin{equation}
  N_\text{w} = \frac{N_\text{r}}{1 - T N_\text{r} } \ .
  \label{eqn:zaelr}
\end{equation}
Aus der Forderung das im Mittel aus der Summe $N_1 + N_2$, als auch wenn beide Präperate $N_{1+2}$ gleichzeitig auf das Geiger-Müller-Zählrohr gerichtet ist, genau so viele Impulse dedektiert werden sollen, ergibt sich die Gleichung
\begin{equation}
  \frac{N_{1+2}}{1-TN_{1+2}} = \frac{N_{1}}{1-TN_{1}} + \frac{N_{2}}{1-TN_{2}} \ .
\end{equation}
Durch umstellen der Gleichung nach $T$ lässt sich anhand der Messgrößen $N_1, N_2, N_{1+2}$  die Totzeit näherungsweise bestimmen.
\begin{equation}
  T ~ \frac{N_1 + N_2 + N_{1+2}}{2 N_1 N_2}
  \label{eqn:T}
\end{equation}


\subsection{Fehlerrechnung}
Sämtliche Fehler und Fehlerfortpflanzungen werden mit Hilfe des Paketes "uncertainties" \cite{uncertainties} aus Python berechnet. Mit ausnahme des Fehlers der Zählrate, dieser wird über
\begin{align*}
	\Delta N = \frac{\sqrt{n_\text{Impulse}}}{\Delta t}
\end{align*}
berechnet. \\
Sämtliche Fits werden mit Hilfe der Paketes "lmfit" \cite{lmfit} aus Python berechnet.
