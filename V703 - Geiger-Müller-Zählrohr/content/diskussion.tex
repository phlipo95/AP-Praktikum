\section{Diskussion}
\label{sec:Diskussion}
Die Steigung des Plateaus der Zählrohr-Charakteristik beträgt
\begin{align*}
  a = (\num{0.09 +- 0.02})\,\text{Vs}
\end{align*}
und die Steigung in \% pro 100\,V beträgt
\begin{align*}
  m = (\num{0.017 +- 0.004})\,\% \ .
\end{align*}

Der Abstand zwischen dem Primär- und dem Nachladeimpuls für 350\,V und 700\,V beträgt
\begin{align*}
  t = (\num{220 +- 50})\,\mu\text{s} \ .
\end{align*}

Die Totzeit, gemessen mit Hilfe des Oszillographen, ist
\begin{align*}
  T_\text{Osz} = (\num{60 +- 1})\,\mu\text{s}
\end{align*}
und über die Zwei-Quellen-Methode ergibt sich ein Wert von
\begin{align*}
  T_\text{2-Q-M} = (\num{20.2 +- 0.4})\,\mu\text{s} \ .
\end{align*}

Die Ladungsmenge die pro Teilchen freigesetzt wird hat einen linearen Zusammenhang mit der Spannung, dieser Zusammenhang ist
\begin{align*}
  Q = (\num{1.94 +- 0.04}) \cdot 10^{-11} \cdot U + (\num{-5.6 +- 0.2}) \cdot 10^{-9} \ .
\end{align*}

Die berechneten Werte besitzten alle einen Fehler, welcher vornehmlich durch Ablesefehler und das zufällige Austreten der Strahlung entsteht.
