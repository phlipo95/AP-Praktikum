\section{Auswertung}
\label{sec:Auswertung}
\subsection{Ablenkung von Elektronen durch das Magnetfeld}
In Tabelle \ref{tab:DIS} sind die Stromstärken $I_\text{S}$ zu den entsprechenden Auslenkungen $D$ aufgetragen. In Klammern dahinter befinden sich die jeweiligen Angaben zur Beschleunigungsspannung.
\begin{table}[H]
  \centering
  \begin{tabular}{c| c c c c c }
    \toprule
    $D / \frac{Zoll}{4}$ & $I$ / A (250 V) & $I$ / A (300 V) & $I$ / A (350 V) & $I$ / A (400 V) & $I$ / A (450 V) \\
    \midrule
    1 &	0.00	&0.00	&0.00	&0.00	&0.00	\\
    2 &	0.27	&0.34	&0.37	&0.42	&0.41	\\
    3 &	0.63	&0.70	&0.74	&0.83	&0.85	\\
    4 &	0.95	&1.05	&1.15	&1.25	&1.30	\\
    5 &	1.27	&1.40	&1.55	&1.61	&1.73	\\
    6 &	1.57	&1.76	&1.91	&2.01	&2.15	\\
    7 &	1.92	&2.11 	&2.27	&2.47	&2.59	\\
    8 &	2.23	&2.45	&2.67	&2.91	&3.02	\\
    9 &	2.53	&2.81	&3.02	&---	&---	\\
    \bottomrule
  \end{tabular}
  \caption{Auslenkung der Elektronen in Abhägigkeit des Spulenstroms bei unterschiedlichen Beschleunigungsspannungen}
  \label{tab:DIS}
\end{table}
Anhand derer lassen sich durch Verwendung von Gleichung \eqref{eqn:BH} die magnetischen Feldstärken errechnen. Diese werden in Diagramm \ref{fig:bfeld1} bis \ref{fig:bfeld5} gegen $D/(L^2+D^2)$ aufgetragen sowie ein linearer Fit der Form 
\begin{equation}
  f(x) = a \cdot x + b
  \label{eqn:fit}
\end{equation}
durch die Messpunkte gelegt.
\begin{figure}[H]
  \centering
  \includegraphics[height=10cm]{B-Feld1.pdf}
  \caption{Diagramm zur Bestimmung der spezifischen Ladung}
  \label{fig:bfeld1}
\end{figure}
\begin{figure}[H]
  \centering
  \includegraphics[height=10cm]{B-Feld2.pdf}
  \caption{Diagramm zur Bestimmung der spezifischen Ladung} 
  \label{fig:bfeld2}
\end{figure}
\begin{figure}[H]
  \centering
  \includegraphics[height=10cm]{B-Feld3.pdf}
  \caption{Diagramm zur Bestimmung der spezifischen Ladung} 
  \label{fig:bfeld3}
\end{figure}
\begin{figure}[H]
  \centering
  \includegraphics[height=10cm]{B-Feld4.pdf}
  \caption{Diagramm zur Bestimmung der spezifischen Ladung} 
  \label{fig:bfeld4}
\end{figure}
\begin{figure}[H]
  \centering
  \includegraphics[height=10cm]{B-Feld5.pdf}
  \caption{Diagramm zur Bestimmung der spezifischen Ladung} 
  \label{fig:bfeld5}
\end{figure}
Aus den Steigungen $a$ der Fits lassen sich durch Umformung der Gleichung \eqref{eqn:e0m} zu,
\begin{equation}
  \frac{e_0}{m_0} = 8 U_\text{B} a^2 
  \label{eqn:e0m0}
\end{equation}
die spezifische Ladung der Elektronen berechnen. Diese sind mit den zugehörigen Beschleunigungspannungen und ihrern Fitparametern in Tabelle \ref{tab:e0m0} aufgetragen. Die Abweichung der gemittelten spezifischen Ladung von dem Literaturwert \cite{e0m0} beträgt 1 \%.
\begin{table}[H]
  \centering
  \begin{tabular}{c| c c c}
    \toprule
    $U_\text{B}$  & $e_0/(m_0 \cdot 10^{11}) $ experimentell & Fitparameter a / $10^3$ & Fitparameter b / $10^{-2}$\\
    \midrule
    250	& (\num{1.77 +- 0.05}) &(\num{9.4 +- 0.1}) & (\num{3 +- 1})	 \\
    300	& (\num{1.76 +- 0.05}) &(\num{8.6 +- 0.1}) & (\num{3 +- 1})	 \\
    350	& (\num{1.76 +- 0.04}) &(\num{7.9 +- 0.1}) & (\num{2 +- 1})	 \\
    400	& (\num{1.78 +- 0.07}) &(\num{7.5 +- 0.1}) & (\num{2 +- 2})	 \\
    450	& (\num{1.80 +- 0.04}) &(\num{7.1 +- 0.1}) & (\num{2 +- 1})	 \\
    \midrule
    --- 	& $\overline{(\num{1.78 +- 0.02})}$ & ---  & --- \\
    %&\multicolumn{2}{l}{= Mittelwert $\pm$ Fehler des Mittelwertes}	\\
    \bottomrule
  \end{tabular}
  \caption{spezifische Ladung und Fitparameter}
  \label{tab:e0m0}
\end{table}
\subsection{Intensität des Erdmagnetfeld}
Für den Versuch wird ein Inklinationswinkel von
\begin{equation}
  \Phi_\text{Inc} = (\num{66 +-3})°
  \label{eqn:Inc}
\end{equation}
gemessen. Der Strom $I_\text{K}$ um das Magnetfeld der Erde mittels Helmholzspulen zu kompensieren beträgt
\begin{equation}
  I_\text{K} = 0.27 \, \text{A} \ .
  \label{IK}
\end{equation}
Mit Formel \eqref{eqn:BH} ergibt sich für den Strom $I_\text{K}$ ein magnetisches Feld von
\begin{equation}
  B_\text{Helm} = 1.72 \cdot 10^{-5} \, \text{T} \ .
  \label{eqn:Binc}
\end{equation}
Unter Berücksichtigung des Inklinationswinkel des magnetischen Feldes ergibt sich eine Totalintesistät von
\begin{equation}
  B_\text{Erd} = \frac{1.72}{\text{cos}(66)} 10^{-5} \, \text{T} = 4.34 \cdot 10^{-5} \, \text{T} \ .
  \label{eqn:Berd}
\end{equation}
\subsection{Empfindlichkeit der Braunschen Röhre}
In einem linearen Diagramm werden die Ablenkspannungen $U_\text{d}$ aus Tabelle \ref{tab:DIS} gegen die Verschiebung $D$ des Leuchtflecks aufgetragen. In den Klammern steht jeweils die Beschleunigungsspannung $U_\text{B}$ in Volt, damit die Messwerte zugeordnet werden können.
\begin{table}[H]
  \centering
  \begin{tabular}{c| c c c c c }
    \toprule
    $D / \frac{Zoll}{4}$ & $U_\text{D}$ / V (180) & $U_\text{D}$ / V (230) & $U_\text{D}$ / V (260) & $U_\text{D}$ /V (300) & $U_\text{D}$ / V (350) \\
    \midrule
    1 &	-20.02	&-25.21	&-28.00	&-32.19	&-28.00	\\
    2 &	-16.12	&-20.24	&-22.99	&-27.04	&-22.99	\\
    3 &	-12.64	&-16.07	&-18.05	&-21.16	&-18.05	\\
    4 &	-8.88	&-12.18	&-12.65	&-15.28	&-12.65	\\
    5 &	-5.10	&-7.59	&-7.51	&-9.58	&-7.51	\\
    6 &	-1.37	&-2.69	&-2.64	&-4.02	&-2.64	\\
    7 &	2.62	&2.04	&2.73	&2.17	&2.73	\\
    8 &	6.53	&6.79	&8.34	&7.97	&8.34	\\
    9 &	10.44	&11.60	&13.57	&14.40	&13.57	\\
    \bottomrule
  \end{tabular}
  \caption{Auslenkung der Elektronen in Abhängigkeit der Ablenkspannung bei unterschiedlichen Beschleunigungsspannungen}
  \label{tab:DIS}
\end{table}
\begin{figure}[H]
  \centering
  \includegraphics[height=8cm]{E-Feld1.pdf}
  \caption{Empfindlichkeit der Braunschen Röhre bei einer Beschleunigungsspannung von 180 V}
  \label{fig:empf1}
\end{figure}
\begin{figure}[H]
  \centering
  \includegraphics[height=8cm]{E-Feld2.pdf}
  \caption{Empfindlichkeit der Braunschen Röhre bei einer Beschleunigungsspannung von 230 V}
  \label{fig:empf2}
\end{figure}
\begin{figure}[H]
  \centering
  \includegraphics[height=8cm]{E-Feld3.pdf}
  \caption{Empfindlichkeit der Braunschen Röhre bei einer Beschleunigungsspannung von 260 V}
  \label{fig:empf3}
\end{figure}
\begin{figure}[H]
  \centering
  \includegraphics[height=8cm]{E-Feld4.pdf}
  \caption{Empfindlichkeit der Braunschen Röhre bei einer Beschleunigungsspannung von 300 V}
  \label{fig:empf4}
\end{figure}
\begin{figure}[H]
  \centering
  \includegraphics[height=8cm]{E-Feld5.pdf}
  \caption{Empfindlichkeit der Braunschen Röhre bei einer Beschleunigungsspannung von 350 V}
  \label{fig:empf5}
\end{figure}

Es wird jeweils ein Fit entsprechend Gleichung \ref{eqn:fit} durch die Messreihe gelegt, anhand derer die Empfindlichkeit der Braunschen Röhre bestimmt wird (siehe Abbildung \ref{fig:empf1} bis \ref{fig:empf5}). Die Fitparameter zu den verschiedenen Beschleunigungsspannungen werden in Tabelle \ref{tab:fit} aufgeführt. 
\begin{table}[H]
  \centering
  \begin{tabular}{c| c c}
    \toprule
    &Fitparameter a / $10^{-3}$ & Fitparameter b / $10^{-2}$\\
  \midrule
    $U_\text{B}$ = 180 V & 1.67	& 3.37 	\\
    $U_\text{B}$ = 230 V & 1.38	& 3.52	\\
    $U_\text{B}$ = 260 V & 1.22	& 3.45	\\
    $U_\text{B}$ = 300 V & 1.09	& 3.57	\\
    $U_\text{B}$ = 350 V & 0.92	& 2.87	\\
  \toprule
  \end{tabular}
  \caption{Fitparameter der Ausgleichsgraden aus Diagramm \ref{fig:empf1} bis \ref{fig:empf5}}
  \label{tab:fit}
\end{table}
Diese werden in einem weiterem Diagramm \ref{fig:K} gegen den Kehrwert der Beschleunigungsspannung aufgetragen.
\begin{figure}
  \centering
  \includegraphics[height=8cm]{E-Feld6.pdf}
  \caption{Bestimmtung der Kenngröße}
  \label{fig:K}
\end{figure}
Anhand der Steigung des weiteren Fits lässt sich die Kenngröße
\begin{equation}
  K_\text{theo} = \frac{p L}{2 d} = \frac{0.019 \, \text{m} \cdot 0.1533 \, \text{m}}{2 \cdot 0.0051 \, \text{m}} = 0.286 \, \frac{1}{\text{m}}
  \label{eqn:Ktheo}
\end{equation}
des Systems bestimmen. Der Fit liefert eine Kenngröße von
\begin{equation}
  K_\text{Messung} = (\num{0.27 +- 0.01}) \, \frac{1}{\text{m}} \ .
  \label{Kmess}
\end{equation}
\subsection{Kathodenstrahl-Oszillograph}
Die gemessene Frequenzverhältnisse sind in Tabelle \ref{tab:sync} aufgeführt.
\begin{table}[H]
  \centering
  \begin{tabular}{c c c}
    \toprule
    Frequenzverhältnis $n$ & $\nu_\text{Messung}$ & $n \cdot \nu_\text{Theoretisch}$ \\
    \midrule
    1:2	& 159.96& 79.98 \\
    1:1	& 79.75 & 79.95	\\
    2:1	& 39.87 & 79.74	\\
    3:1	& 26.65 & 79.95	\\
    \bottomrule
  \end{tabular}
  \caption{Synchronisationsfrequenzen}
  \label{tab:sync}
\end{table}
Ein Bild einer stehenden Welle ist in Abbildung \ref{fig:pic} zu sehen.
\begin{figure}[H]
  \centering
  \includegraphics[height=7cm]{picture/sinus.png}
  \caption{Stehende Sinuswelle}
  \label{fig:pic}
\end{figure}
Aus Mittelung der mit dem Frequenzverhältnis multiplizierten gemessenen Frequenzen ergibt sich eine mittlere Frequenz von
\begin{equation}
  \overline{\nu_\text{Messung}} = \num{79.91 +- 0.05} \text{Hz}
  \label{eqn:numess}
\end{equation}
Desweiteren soll der Schwellwert der Spannung $U_\text{B}$ werden. Da die Empfindlichkeit der Röhre bereits aus den Fitparametern aus der Tabelle \ref{tab:fit} bekannt ist, ergibt dieser sich durch Umformen der Gleichung \eqref{eqn:DUd} nach $U_d$ zu
\begin{equation}
  U_\text{d} = \frac{D 2 d U_\text{B}}{p L} = \frac{D}{E} \ .
  \label{eqn:dfs}
\end{equation}
Für die Beschleunigungsspannung von 260 V ergibt sich für die verschiedenen Verschiebungen $D$ auf dem Schirm die in der Tabelle \ref{tab:scheitel} aufgeführten Scheitelwerte.
\begin{table}[H]
  \centering
  \begin{tabular}{c c c}
    \toprule
	Frequenzverhältniss & D / mm & $U_\text{d}$ / V \\
    \midrule
	1:2	& 6.3 & 3.8 	\\
	1:1	& 6.3 & 3.8	\\
	2:1	& 5.7 & 3.4	\\
	3:1	& 5.7 & 3.4	\\
    \bottomrule
  \end{tabular}
  \caption{Verschiebung und Scheitelwert der Spannung}
  \label{tab:scheitel}
\end{table}
