\section{Diskussion}
\label{sec:Diskussion}

Die berechneten Exponenten aus dem Kapitel (\ref{sec:K}) weichen von dem Referenzwert $x = 1.5$, aus der Gleichung (\ref{eqn:jLS}), um
\begin{align*}
  & x_1 = \num{1.379 +- 0.001} \\
  \text{Abweichung:}\ & \frac{\vert x - x_1 \vert }{x} \cdot 100 = 8.07\, \% \\
  & x_2 = \num{1.345 +- 0.002} \\
  \text{Abweichung:}\ & \frac{\vert x - x_2 \vert}{x} \cdot 100 = 10.33\, \%
\end{align*}
ab. \\
Die Kathodentemperatur aus dem Kapiteln (\ref{sec:A}) beträgt
\begin{align*}
  T = (\num{2700 +- 80}) \, \text{K}
\end{align*}
und aus dem Kapitel (\ref{sec:L}) betragen die Kathodentemperaturen
\begin{align*}
  T_1 = 2310\, \text{K} \\
  T_2 = 2265\, \text{K} \\
  T_3 = 2051\, \text{K} \\
  T_4 = 1951\, \text{K} \\
  T_5 = 1870\, \text{K} \\
\end{align*}
Die Kathodentemperaturen weichen zwischen $\frac{\vert T - T_1 \vert}{T} \cdot 100 = 14.44\,\%$ und $\frac{\vert T - T_2 \vert}{T} \cdot 100 = 30.74\,\%$ voneinander ab.\\
Der errechnete Mittelwert der Austrittsarbeit für Wolfram aus Kapitel \eqref{sec:AfW} beträgt ($\num{4.78 +- 0.08}$) eV und weicht damit um 5.33\,\% von dem Literaturwert ab. Der Literaturwert für die Austrittsarbeit beträgt 4.54 eV (\cite{Wolfram}).
