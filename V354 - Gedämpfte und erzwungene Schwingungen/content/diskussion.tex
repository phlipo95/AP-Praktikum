<<<<<<< HEAD
\section{Diskussion}
\label{sec:Diskussion}
Bei allen Experimenten wird die Länge des Koaxialkabels vernachlässigt, da der Phasenunterschied bei Messgrößen im kHz Bereich vernachlässigbar gering ist.
\subsection{Zeitabhänigkeit der Amplitude und Dämpfungswiderstand einer gedämpften Schwingung}
Aufgrund eines Rauschens des Signals welches auf das Oszilloskop ausgegeben wird, kommt es dazu, dass der Fit die Spannungspeaks nicht immer genau trifft. Als Konsequenz dessen sind die Fehler relativ (f =  ~10 \%) groß und die abgeleiteten Größen infolge dessen auch Fehlerbehaftet.
\subsection{Frequenzabhängigkeit der Kondensatorspannung eines
Serienresonanzkreis}
Auffällig ist das alle bis auf 3 Messwerteunterhalb von 10 kHz oberhalb der theoretischen Spannung liegen. Aufgrund der ungenauigkeit beim ablesen der Resonanzfrequenz, als auch der ``Halbwertszeiten'' kommt es zu Fehlern von 3 und 6.9 \%. Dabei ist die Messung welche über die Resonanzfrequenz erfolgt genauer, da es einfacher ist das Maximum dem Diagramm zu entnehmen als ein Halbwertsbreite exakt abzulesen.
Alle experimentell ermittelten Messwerte liegen in etwa im Toleranzbereich der Messgrößen, bis auf die Güte. Es scheint so, als ob ein systematischer Fehler vorliegt. Ausschließlich bei f $\rightarrow \infty $ und f $\rightarrow 0$ wird der Erwatungswert erfüllt.
\subsection{Frequenzabhängigkeit der Phase zwischen Erreger- und
Kondensatorspannung}
Unter Betrachtung der Abbildung 10 fällt auf, dass der Wert bei ~40 kHz aus der Messreihe sticht. Scheinbar handelt es sich dabei um einen Messfehler. Auch hier liegen die Werte bis 10 kHz nicht genau auf dem Fit. Eine  mögliche Ursache dafür ist, dass der arctan erst im Grenzwert - $\nu$ $\rightarrow$ 0 strebt. Der Fehler der Fitfunktion hat zu Folge, dass die berechnete Güte die größte Messunsicherheit aller experimentell bestimmten Güten hat. Jedoch weicht der Mittelwert lediglich um 4.5 \% ab.
\subsection{Vergleich der Messmethoden}
Ein Vergleich der drei Messmethoden ist nicht direkt möglich. Zwei der drei Methoden beruhen auf dem ähnlichen System aus der Resonanzfrequenz und den ``Halbwertszeiten'' die Güte zu bestimmen. ????Lediglich die Methode mittels Resonanzfrequenz die Güte unterscheidet sich von diesen. Sie scheint am praktikabelsten für eine schnelle Messung geeignet zu sein, jedoch müssen auch bei dieser Methode die Kapazität als auch der Widerstand bekannt sein. Im Umkehrschluss ist es Vorteilhaft eine der anderen beiden Methoden zu benutzen falls diese unbekannt sind. Ein möglicher Nachteil der Methode welche auf der Phasenverschiebung beruht, gegenüber der mittels der Kondensatorspannung ist, dass für diesen ein Zwei-Kanal-Oszilloskop benötigt wird wohingegen bei der anderen ein Ein-Kanal-Oszilloskop ausreicht.
=======
\section{Diskussion}
\label{sec:Diskussion}
Bei allen Experimenten wird die Länge des Koaxialkabels vernachlässigt, da der Phasenunterschied bei Messgrößen im kHz Bereich vernachlässigbar gering ist.
\subsection{Zeitabhänigkeit der Amplitude und Dämpfungswiderstand einer gedämpften Schwingung}
Aufgrund eines Rauschens des Signals welches auf das Oszilloskop ausgegeben wird, kommt es dazu, dass der Fit die Spannungspeaks nicht immer genau trifft. Als Konsequenz dessen sind die Fehler relativ (f =  ~10 \%) groß und die abgeleiteten Größen infolge dessen auch Fehlerbehaftet.
\subsection{Frequenzabhängigkeit der Kondensatorspannung eines
Serienresonanzkreis}
Auffällig ist das alle bis auf 3 Messwerteunterhalb von 10 kHz oberhalb der theoretischen Spannung liegen. Aufgrund der ungenauigkeit beim ablesen der Resonanzfrequenz, als auch der ``Halbwertszeiten'' kommt es zu Fehlern von 3 und 6.9 \%. Dabei ist die Messung welche über die Resonanzfrequenz erfolgt genauer, da es einfacher ist das Maximum dem Diagramm zu entnehmen als ein Halbwertsbreite exakt abzulesen.
Alle experimentell ermittelten Messwerte liegen in etwa im Toleranzbereich der Messgrößen, bis auf die Güte. Es scheint so, als ob ein systematischer Fehler vorliegt. Ausschließlich bei f $\rightarrow \infty $ und f $\rightarrow 0$ wird der Erwatungswert erfüllt.
\subsection{Frequenzabhängigkeit der Phase zwischen Erreger- und
Kondensatorspannung}
Unter Betrachtung der Abbildung 10 fällt auf, dass der Wert bei ~40 kHz aus der Messreihe sticht. Scheinbar handelt es sich dabei um einen Messfehler. Auch hier liegen die Werte bis 10 kHz nicht genau auf dem Fit. Eine  mögliche Ursache dafür ist, dass der arctan erst im Grenzwert - $\nu$ $\rightarrow$ 0 strebt. Der Fehler der Fitfunktion hat zu Folge, dass die berechnete Güte die größte Messunsicherheit aller experimentell bestimmten Güten hat. Jedoch weicht der Mittelwert lediglich um 4.5 \% ab.
\subsection{Vergleich der Messmethoden}
Ein Vergleich der drei Messmethoden ist nicht direkt möglich. Zwei der drei Methoden beruhen auf dem ähnlichen System aus der Resonanzfrequenz und den ``Halbwertsbreiten'' die Güte zu bestimmen. Lediglich die Methode die Güte aus dem Kehrwert des Produkt des Widerstandes, der Kapazität und der Resonanzfrequenz zu berechnen unterscheidet sich von diesen. Sie scheint am praktikabelsten für eine schnelle Messung geeignet zu sein, jedoch müssen auch bei dieser Methode die Kapazität als auch der Widerstand bekannt sein. Im Umkehrschluss ist es Vorteilhaft eine der anderen beiden Methoden zu benutzen falls diese unbekannt sind. Ein möglicher Nachteil der Methode welche auf der Phasenverschiebung beruht, gegenüber der mittels der Kondensatorspannung ist, dass für diesen ein Zwei-Kanal-Oszilloskop benötigt wird wohingegen bei der anderen ein Ein-Kanal-Oszilloskop ausreicht.
>>>>>>> e4f1649d517f5f97fc10e170ed3203910a977da5
