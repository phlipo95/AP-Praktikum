\section{Diskussion}
\label{sec:Diskussion}
Der experimentell ermittelte Abschirmkoeffizient für Blei beträgt ($\num{106 +- 2}$) / m. Verglichen mit dem Compton-Abschirmkoeffizient ist dieser um $ ~ 35 \% $ größer. Dies liegt womöglich daran das die Gammaquanten zusätzlich zum Compton-Effekt auch durch den inneren Photoeffekt bei Blei abgeschirmt werden. 
Die gemittelte Angangsaktivität beträgt (\num{141 +- 12}) zerfälle die Sekunde. Dabei weicht der Wert von Kupfer um 5 \% nach unten und der von Blei um 5 \% nach oben ab. 
Die Maximale Massenbelegung von dem $\beta$-Strahler in Aluminium beträgt 0.65 kg / $\text{m}^2$. Die Gesamtenergie des Strahlers ist $\num{0.26 +- 0.02}$ MeV was einer Abweichung vom Theoriewert (0.294 \cite{wiki}) von 11 \% entspricht. 
