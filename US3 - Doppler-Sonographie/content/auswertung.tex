\section{Auswertung}
\label{sec:Auswertung}

\subsection{Strömungsgeschwindigkeit in Abhängigkeit vom Dopplerwinkel}
\label{sec:asd}
In der Tabelle \eqref{tab:Dopplerwinkel} sind die Dopplerwinkel für die Winkel des Prismas aufgelistet.

\begin{table}[H] %Dopplerwinkel
  \centering
  \begin{tabular}{c|c|c}
    \hline
    Prismenwinkel / Grad & Dopplerwinkel / Grad & $\cos(\alpha)$ \\
    \hline
    15 & 80.06 & 0.17 \\
    30 & 70.53 & 0.33 \\
    60 & 54.74 & 0.58 \\
  \end{tabular}
  \caption{Die Dopplerwinkel und der $\cos$ davon}
  \label{tab:Dopplerwinkel}
\end{table}

Mit diesen Dopplerwinkelen lässt sich nun $\Delta \nu / \cos(\alpha)$ gegen $v$ in einem Diagramm auftragen. In Abbildung \eqref{fig:Doppler1} ist der Fit für das dünnste Rohr mit einem innen Durchmesser von 7\,mm. In Abbildung \eqref{fig:Doppler2} ist der Fit für das mittlere Rohr mit einem innen Durchmesser von 10\,mm. In Abbildung \eqref{fig:Doppler3} ist der Fit für das dickste Rohr mit einem innen Durchmesser von 16\,mm. \\
Aus den drei Abbildungen lässt sich erkennen, dass die Frequenzverschiebung linear von der Geschwindigkeit abhängig ist. Die Linearität ist durch die Fits einer linearen Regression gezeigt. Des Weiteren kann den Abbildungen entnommen werden, dass je kleiner der Prismenwinkel, desto stärker steigt die Frequenzverschiebung an.

\begin{figure}[H]
  \centering
  \includegraphics[height=8cm]{build/Doppler1.pdf}
  \caption{Plot für das dünne Rohr (7mm \O).}
  \label{fig:Doppler1}
\end{figure}

\begin{figure}[H]
  \centering
  \includegraphics[height=8cm]{build/Doppler2.pdf}
  \caption{Plot für das dünne Rohr (10mm \O).}
  \label{fig:Doppler2}
\end{figure}

\begin{figure}[H]
  \centering
  \includegraphics[height=8cm]{build/Doppler3.pdf}
  \caption{Plot für das dünne Rohr (16mm \O).}
  \label{fig:Doppler3}
\end{figure}

\subsection{Strömungsprofil der Doppler-Flüssigkeit}
\label{sec:asdf}
Die gemessene Streuintensität ist in Abhängigkeit von der Tiefe der Messung ist in Abbildung \eqref{fig:Int} zu sehen.

\begin{figure}[H]
  \centering
  \includegraphics[height=7cm]{build/Intensitaet.pdf}
  \caption{Plot der Intensität gegen die Tiefe.}
  \label{fig:Int}
\end{figure}

Die Geschwindigkeit der Dopplerflüssigkeit ist in Abhängigkeit von der Tiefe der Messung ist in Abbildung \eqref{fig:Gesch} zu sehen.

\begin{figure}[H]
  \centering
  \includegraphics[height=7cm]{build/Geschwindigkeit.pdf}
  \caption{Plot der Geschwindigkeit gegen die Tiefe.}
  \label{fig:Gesch}
\end{figure}












%
