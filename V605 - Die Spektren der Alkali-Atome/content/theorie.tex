\section{Theoretische Grundlage}
\label{sec:Theorie}
Ziel des Versuches ist es die innere Abschirmungszahlen verschiedener Alkali-Metalle anhand deren Emissionslinien zu bestimmen.
Alkali-Metalle besitzten komplett besetzte Schalen und nur ein Valenzelektron. Diese lassen sich mit Hilfe der Ein-Elektronen-Näherung recht einfach beschreiben. Der Anatz der Näherung beruht darauf, dass die Ladungen der Elektronen, welche die kompletten Schalen besetzen sich mit denen der Protonen des Kerns kompensieren. Somit kann das Atom als Wasserstoffatom genähert werden. Das zwischen dem Proton und Elektron liegende Columbfeld, wird durch die inneren Elektronen abgeschirmt. Dies wird im weiteren Verlauf der Rechnung durch die Abschirmungszahl $\sigma$ berücksichtigt.
\begin{equation}
  z_\text{eff} = z - \sigma
  \label{eqn:zeff}
\end{equation}
Somit errechnet sich die effektive Anzahl von Protonen entsprechend Gleichung \eqref{eqn:zeff}.
Um die Energieeigenwerte des Symstems zu berechnen wird zunächst die Schrödingergleichung aufgestellt,
\begin{equation}
  \left( \sum_\text{i} \frac{P_\text{i}^2}{2 m_\text{i}} + U \right) \Psi = E\, \Psi
  \label{eqn:Sch}
\end{equation}
für die sich die Lösungen
\begin{equation}
  E(n) = R_{\infty} \frac{1}{n^2}, \ n = 1, \, 2, \, \cdots
  \label{eqn:En}
\end{equation}
ergeben. Dabei ist $n$ die Hauptquantenzahl. Da es sich jedoch um ein Kugelsymetrisches Problem handelt muss zunächst noch der Laplaceoperator in Kugelkoordinaten eingeführt werden. Zusätzlich wird nun beim Potential die Abschirmkonstante berücksichtigt, so dass sich das Potential der Form
\begin{equation}
  U = - \frac{\left( z - \sigma \right) e_0^2}{4 \pi  \varepsilon_0 r}
  \label{schrö}
\end{equation}
ergibt. Die Lösung der für das an das System angepasste Schrödingergleichung wird relativistisch betrachtet und getaylort, so dass sich die Energieeigenwerte
\begin{equation}
  E_\text{rel n,l} = -R_{\infty} \left( \frac{(z - \Sigma)^2}{n²} + \alpha^2 \frac{(z - \sigma)^4}{n^3}\left( \frac{2}{2 l +1} - \frac{3}{4n} \right) \right)
  \label{eqn:ham}
\end{equation}
ergeben. Dabei entspricht $\alpha$ der Sommerfeldschen Feinstrukturkonstante, $R_{\infty}$ der Rydbergenergie
\begin{equation}
  \alpha := \frac{e_0^2}{2 h c \varepsilon_0}
  \label{eqn:alpha}
\end{equation}
und $l$ ist die Bahndrehquantenzahl die beim lösen der Schrödingergleichung aus der Kugelflächenfunktion resultiert. Desweiteren muss die Spin-Bahn-Kopplung $S$ der Elektronen berücksichtigt werden. Sie ist neben dem aus den kugelsymetrischen Laplaceoperator folgender Drehimpuls $L$ ein weiterer Teil des Gesamtdrehimpulses. Mittels Störungstheorie wird der Einfluss des Spins berechnet. Es ergibt sich die Energieeigenwertgleichung
\begin{equation}
  E_\text{n,j} = -R_{\infty}\left( \frac{(z - \sigma)^2}{n²} + \alpha^2 \frac{(z - \sigma)^4}{n^3} \left( \frac{1}{j + 0.5} - \frac{3}{4n} \right) \right)
  \label{<++>}
\end{equation}
wobei der Gesamtdrehimpuls $j$ nur die Werte $l + \frac{1}{2}$ und $l - \frac{1}{2}$ annehmen kann.

Angeregte Valenzelektronen haben bei den durch die "Quantenzahlen $n ,l$ und $j$ gekennzeichneten Energieniveaus verschiedene Übergangswahrscheinlichkeiten" \cite{sample}. Die entsprechende Bahndrehquantenzahl $l$ ändert sich jeweils um $\Delta l$ = $\pm 1$ . Darüber hinaus ist ein Übergang der Spinquantenzahl $\Delta j = 0$ zwar möglich, aber sehr unwahrscheinlich. Die Hauptquantenzahl $n$ ist nicht eingeschränkt, wird jedoch mit steigendem $n$ immer unwahrscheinlicher. Das Phänomen das bei gleicher Bahndrehquantenzahl und unterschiedlichen Spinquantenzahlen bei Spektrallinien dicht beieinadner liegen, wird als Dublett-Struktur bezeichnet. In Abbildung \ref{fig:ene} ist ein Beispiel für verschiedene Energieniveaus abgebildet.

\begin{figure}
  \centering
  \includegraphics[height=6cm]{picture/eniv.png}
  \caption{Beispiel für Energieniveaus der Spektrallinien. \cite{sample}}
  \label{fig:ene}
\end{figure}

Um Schlussendlich die Energie des Leuchtelektron genau zu bestimmen wird die Abschirmkonstante in zwei Teile eingeteilt. Einerseits in die Konstante der vollen Abschirmung $\sigma_1$ und der der inneren Abschirmung $\sigma_2$. Da die zweite Abschirmungskonstante $\sigma_2$ ausschließlich die innere Abschirmung beschreibt, ist diese nur geringfügig kleiner, als die volle Abschirmungskonstante $\sigma_1$ welche die Abschirmung aller Elektronen  impliziert. Als Dublett wird ein Quantenmechanischer Zustand bezeichnet bei dem sich lediglich die Quantenzahl $j$ von dem des anderen Zustandes um 1 unterscheidet. Anhand der Energiedifferenz $\Delta E_\text{D}$ von Dubletts lässt sich die Abschirmungszahl $\sigma_2$ mit Gleichung \eqref{eqn:ded} bestimmen,
\begin{equation}
  \Delta E_\text{D} = \frac{R_{\infty} \alpha^2}{n^3} \left( z - \sigma_2 \right)^4 \frac{1}{l(l+1)}
  \label{eqn:ded}
\end{equation}
indem die Gleichung nach der Abschirmkonstante aufgelöst wird. Die Energiedifferenz wird anhand des Wellenlängenunterschiedes der beiden Spektrallinien des Dubletts entsprechend der Gleichung
\begin{equation}
  \Delta E_\text{D} = hc\left( \frac{1}{\lambda} - \frac{1}{\lambda'} \right)
  \label{eqn:ed}
\end{equation}
bestimmt.
Mit Hilfe der Interferenz an einem Gitter werden die Spektrallinien in ihre einzelnen Wellenlängen aufgespalten. Konstruktive Interferenz wird bei den Maximums $k$ter Ordung unter dem Winkel $\varphi$ beobachtet. Die Gitterkonstante wird als $g$ bezeichnet.
\begin{equation}
  \sin \, \varphi = k \frac{\lambda}{g}
  \label{eqn:phi}
\end{equation}
Für die Bestimmung des Wellenlängenunterschieds der Dubletts muss das Okularmikrometer geeicht werden. Näheres zu dem Verfahren wird in der Durchführung erwähnt. Aus Trigonometrischen Überlegungegen ergibt sich die Formel
\begin{equation}
  \Delta\lambda = \frac{\cos \, \varphi}{\cos \, \varphi_{1,2}} \frac{\Delta s}{\Delta t}\left( \lambda_1 - \lambda_2 \right) \ .
  \label{eqn:dlam}
\end{equation}
$\varphi$ entspricht dem gemittelten Winkel der Doublettlinien und $\varphi_{1,2}$ die gemittelte Differenz der Doublettlinien. $\Delta s$ ist der Abstand zwischen den beiden Doublettlinien und $\Delta t$ der, der beiden bekannten Spektrallinien $\lambda_1$ und $\lambda_2$.

\subsection{Fehlerrechnung}
Sämtliche Fehlerrechnungen werden mit Hilfe von Python 3.4.3 durchgeführt.
\subsubsection{Mittelwert}
Der Mittelwert einer Messreihe $x_\text{1}, ... ,x_\text{n}$ lässt sich durch die Formel
\begin{equation}
	\overline{x} = \frac{1}{N} \sum_{\text{k}=1}^\text{N} x_k
	\label{eqn:ave}
\end{equation}
berechnen. Der Fehler des Mittelwertes beträgt
\begin{equation}
	\Delta \overline{x} = \sqrt{ \frac{1}{N(N-1)} \sum_{\text{k}=1}^\text{N} (x_\text{k} - \overline{x})^2}
	\label{eqn:std}
\end{equation}

\subsubsection{Gauß'sche Fehlerfortpflanzung}
Wenn $x_\text{1}, ..., x_\text{n}$ fehlerbehaftete Messgrößen im weiteren Verlauf benutzt werden, wird der neue Fehler $\Delta f$ mit Hilfe der Gaußschen Fehlerfortpflanzung angegeben.
\begin{equation}
	\Delta f = \sqrt{\sum_{\text{k}=1}^\text{N} \left( \frac{ \partial f}{\partial x_\text{k}} \right) ^2 \cdot (\Delta x_\text{k})^2}
	\label{eqn:var}
\end{equation}

\subsubsection{Lineare Regression}
Die Steigung und y-Achsenabschnitt einer Ausgleichsgeraden werden gegebenfalls mittels Linearen Regression berechnet.
\begin{equation}
	y = m \cdot x + b
	\label{eqn:reg}
\end{equation}
\begin{equation}
	m = \frac{ \overline{xy} - \overline{x} \overline{y} } {\overline{x^2} - \overline{x}^2}
	\label{eqn:reg_m}
\end{equation}
\begin{equation}
	b = \frac{ \overline{x^2}\overline{y} - \overline{x} \, \overline{xy}} { \overline{x^2} - \overline{x}^2}
	\label{eqn:reg_b}
\end{equation}
